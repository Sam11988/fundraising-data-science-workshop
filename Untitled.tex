\documentclass[]{article}
\usepackage{lmodern}
\usepackage{amssymb,amsmath}
\usepackage{ifxetex,ifluatex}
\usepackage{fixltx2e} % provides \textsubscript
\ifnum 0\ifxetex 1\fi\ifluatex 1\fi=0 % if pdftex
  \usepackage[T1]{fontenc}
  \usepackage[utf8]{inputenc}
\else % if luatex or xelatex
  \ifxetex
    \usepackage{mathspec}
  \else
    \usepackage{fontspec}
  \fi
  \defaultfontfeatures{Ligatures=TeX,Scale=MatchLowercase}
\fi
% use upquote if available, for straight quotes in verbatim environments
\IfFileExists{upquote.sty}{\usepackage{upquote}}{}
% use microtype if available
\IfFileExists{microtype.sty}{%
\usepackage{microtype}
\UseMicrotypeSet[protrusion]{basicmath} % disable protrusion for tt fonts
}{}
\usepackage[margin=1in]{geometry}
\usepackage{hyperref}
\hypersetup{unicode=true,
            pdftitle={Untitled},
            pdfborder={0 0 0},
            breaklinks=true}
\urlstyle{same}  % don't use monospace font for urls
\usepackage{longtable,booktabs}
\usepackage{graphicx,grffile}
\makeatletter
\def\maxwidth{\ifdim\Gin@nat@width>\linewidth\linewidth\else\Gin@nat@width\fi}
\def\maxheight{\ifdim\Gin@nat@height>\textheight\textheight\else\Gin@nat@height\fi}
\makeatother
% Scale images if necessary, so that they will not overflow the page
% margins by default, and it is still possible to overwrite the defaults
% using explicit options in \includegraphics[width, height, ...]{}
\setkeys{Gin}{width=\maxwidth,height=\maxheight,keepaspectratio}
\IfFileExists{parskip.sty}{%
\usepackage{parskip}
}{% else
\setlength{\parindent}{0pt}
\setlength{\parskip}{6pt plus 2pt minus 1pt}
}
\setlength{\emergencystretch}{3em}  % prevent overfull lines
\providecommand{\tightlist}{%
  \setlength{\itemsep}{0pt}\setlength{\parskip}{0pt}}
\setcounter{secnumdepth}{0}
% Redefines (sub)paragraphs to behave more like sections
\ifx\paragraph\undefined\else
\let\oldparagraph\paragraph
\renewcommand{\paragraph}[1]{\oldparagraph{#1}\mbox{}}
\fi
\ifx\subparagraph\undefined\else
\let\oldsubparagraph\subparagraph
\renewcommand{\subparagraph}[1]{\oldsubparagraph{#1}\mbox{}}
\fi

%%% Use protect on footnotes to avoid problems with footnotes in titles
\let\rmarkdownfootnote\footnote%
\def\footnote{\protect\rmarkdownfootnote}

%%% Change title format to be more compact
\usepackage{titling}

% Create subtitle command for use in maketitle
\newcommand{\subtitle}[1]{
  \posttitle{
    \begin{center}\large#1\end{center}
    }
}

\setlength{\droptitle}{-2em}

  \title{Untitled}
    \pretitle{\vspace{\droptitle}\centering\huge}
  \posttitle{\par}
    \author{}
    \preauthor{}\postauthor{}
    \date{}
    \predate{}\postdate{}
  

\begin{document}
\maketitle

\subsection{R Markdown}\label{r-markdown}

This is an R Markdown document. Markdown is a simple formatting syntax
for authoring HTML, PDF, and MS Word documents. For more details on
using R Markdown see \url{http://rmarkdown.rstudio.com}.

When you click the \textbf{Knit} button a document will be generated
that includes both content as well as the output of any embedded R code
chunks within the document. You can embed an R code chunk like this:

\subsection{Including Plots}\label{including-plots}

You can also embed plots, for example:

\begin{verbatim}
## Warning: package 'knitr' was built under R version 3.4.3
\end{verbatim}

\begin{verbatim}
## -- Attaching packages ---------
\end{verbatim}

\begin{verbatim}
## √ ggplot2 3.0.0     √ purrr   0.2.5
## √ tibble  1.4.2     √ dplyr   0.7.8
## √ tidyr   0.8.2     √ stringr 1.3.1
## √ readr   1.3.1     √ forcats 0.2.0
\end{verbatim}

\begin{verbatim}
## Warning: package 'ggplot2' was built under R version 3.4.4
\end{verbatim}

\begin{verbatim}
## Warning: package 'tibble' was built under R version 3.4.3
\end{verbatim}

\begin{verbatim}
## Warning: package 'tidyr' was built under R version 3.4.4
\end{verbatim}

\begin{verbatim}
## Warning: package 'readr' was built under R version 3.4.4
\end{verbatim}

\begin{verbatim}
## Warning: package 'purrr' was built under R version 3.4.4
\end{verbatim}

\begin{verbatim}
## Warning: package 'dplyr' was built under R version 3.4.4
\end{verbatim}

\begin{verbatim}
## Warning: package 'stringr' was built under R version 3.4.4
\end{verbatim}

\begin{verbatim}
## -- Conflicts ------------------
## x dplyr::filter() masks stats::filter()
## x dplyr::lag()    masks stats::lag()
\end{verbatim}

\begin{verbatim}
## Warning: package 'bindrcpp' was built under R version 3.4.4
\end{verbatim}

\begin{longtable}[]{@{}rrr@{}}
\toprule
mpg & wt & hp\tabularnewline
\midrule
\endhead
33.9 & 1.835 & 65\tabularnewline
32.4 & 2.200 & 66\tabularnewline
30.4 & 1.615 & 52\tabularnewline
30.4 & 1.513 & 113\tabularnewline
27.3 & 1.935 & 66\tabularnewline
26.0 & 2.140 & 91\tabularnewline
24.4 & 3.190 & 62\tabularnewline
22.8 & 2.320 & 93\tabularnewline
22.8 & 3.150 & 95\tabularnewline
21.5 & 2.465 & 97\tabularnewline
21.4 & 3.215 & 110\tabularnewline
21.4 & 2.780 & 109\tabularnewline
21.0 & 2.620 & 110\tabularnewline
21.0 & 2.875 & 110\tabularnewline
19.7 & 2.770 & 175\tabularnewline
19.2 & 3.440 & 123\tabularnewline
19.2 & 3.845 & 175\tabularnewline
18.7 & 3.440 & 175\tabularnewline
18.1 & 3.460 & 105\tabularnewline
17.8 & 3.440 & 123\tabularnewline
17.3 & 3.730 & 180\tabularnewline
16.4 & 4.070 & 180\tabularnewline
15.8 & 3.170 & 264\tabularnewline
15.5 & 3.520 & 150\tabularnewline
15.2 & 3.780 & 180\tabularnewline
15.2 & 3.435 & 150\tabularnewline
15.0 & 3.570 & 335\tabularnewline
14.7 & 5.345 & 230\tabularnewline
14.3 & 3.570 & 245\tabularnewline
13.3 & 3.840 & 245\tabularnewline
10.4 & 5.250 & 205\tabularnewline
10.4 & 5.424 & 215\tabularnewline
\bottomrule
\end{longtable}

Note that the \texttt{echo\ =\ FALSE} parameter was added to the code
chunk to prevent printing of the R code that generated the plot.


\end{document}
